Das Thema Datenschutz ist sehr komplex, insbesondere wenn verschiedene Auffassungen und Gewichtungen aufeinandertreffen und in Einklang gebracht werden müssen. Die durch den Europäischen Gerichtshof für nichtig erklärten Abkommen \glqq Safe Harbor\grqq{} und \glqq Privacy Shield\grqq{} mit den USA zeigen auf, dass dieser Einklang sehr schwer, wenn nicht sogar unmöglich zu erreichen ist. Immerhin ist es in den letzten 20 Jahren nicht gelungen, ein solches rechtskonformes Abkommen zu formulieren und auszuhandeln.

Bezüglich aus dem Hause Microsoft stammende Software zeigt sich ebenfalls die unterschiedliche Auffassung, was den Datenschutz anbelangt. Als US-amerikanisches Unternehmen entwickelt Microsoft seine Produkte mit Blick auf die eigenen Datenschutzgrundsätze, welche längst nicht den europäischen Ansprüchen entsprechen und immer wieder für Diskrepanz sorgen. Gerade, wenn neue Produkte eingeführt werden, ist diese besonders groß und es entsteht der Eindruck, dass seitens des Herstellers erst einmal abgetastet wird, wie weit er bezüglich der Datenerhebung gehen kann und es erfolgt eine langsame Annäherung an europäische Standards.

Allerdings ist auch ein Einlenken durch Microsoft zu erkennen, denn immerhin werden Funktionen wie das Sammeln von Betriebssystem-Telemetriedaten oder das Anonymisieren personenbezogener Daten nachgebessert - Dinge, die nach europäischer Sicht auf den Datenschutz wiederum Standard sein sollten. Weiter ist auch zu erwähnen, dass dies oft erst nach Ausübung von Druck seitens Aktivisten und Datenschutzbeauftragten der jeweiligen Länder geschieht und nur so weit wie nötig umgesetzt wird.

Alternativen zu Microsoft-Produkten und auch Großprojekte zur Umstellung ganzer behördlicher Infrastrukturen auf freie Software wurden zwar schon unternommen und werden auch immer wieder angestrebt - aktuell etwa in Schleswig-Holstein oder Mecklenburg-Vorpommern - das LiMux-Projekt zeigte allerdings auch, das es solche Projekte ob ihrer wirtschaftlichen Relevanz und nicht zuletzt der Akzeptanz der Nutzer schwer haben, sich zu behaupten.

Abschließend komme ich zu dem Schluss, dass Unternehmen und auch jeder Einzelne für die Problematik des Datenschutzes sensibilisiert werden müssen beziehungsweise dieses Thema präsenter diskutiert werden muss. Es genügt nicht, einzelne Missstände aufzuzeigen. Vielmehr muss auch klargemacht werden, wie jeder selbst verantwortungsvoll mit den eigenen persönlichen Daten umgehen sollte.
\\