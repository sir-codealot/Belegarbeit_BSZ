\subsection{Alternative Produkte}
Zwar ist Microsoft mit seiner Produktfamilie von Betriebssystemen bis Büro-Standardsoftware klarer Marktführer und augenscheinlich ohne wirkliche Konkurrenz, jedoch gibt es bis zu einem gewissen Grad auch ernstzunehmende Alternativen. So gibt es für nahezu jedes Microsoft-Programm ein Gegenstück freier Software. Unter diesem Sammelbegriff versteht man frei zugängliche, meist kostenlose Software von der auch der Quellcode veröffentlicht ist und welche auch sowohl privat als auch kommerziell verwendet werden darf.\\
Beispiele wären hier \glqq LibreOffice\grqq{}, welches neben einer Textverarbeitung auch eine Tabellenkalkulation und Präsentations-Software mit sich bringt und somit die 3 gängigsten Office-Programme \glqq Word\grqq{}, \glqq Excel\grqq{} und \glqq Powerpoint \grqq{} ersetzt. Den Email-Client \glqq Outlook\grqq{} kann durch Mozillas \glqq Thunderbird\grqq{} abgelöst werden und Microsofts Browser \glqq Internet Explorer\grqq{} und \glqq Edge\grqq{} finden in Form von \glqq Firefox\grqq{} oder \glqq Chromium\grqq{} \footnote{Basis von Googles Chrome Browser, komplett frei von proprietärer Software} adäquaten Ersatz. Will man den Weg zuende gehen kann man sich sogar für ein freies Betriebssystem in Form einer der unzähligen Linux-Derivate entscheiden. Die bekanntesten und ausgereiftesten Vertreter für den Produktiveinsatz stellen hier \glqq Red Hat Enterprise Linux\grqq{} von Red Hat bzw. dessen Community-Variante \glqq Fedora Linux\grqq{} oder auch \glqq Ubuntu\grqq{} von der Canonical Foundation dar. Red Hat und Canonical bieten einen bezahlten Support für Ihre Betriebssysteme an. Prinzipiell kann man diese aber auch ohne nutzen, dies bringt allerdings auch beim Auftreten von Problemen einen gewissen Mehraufwand mit sich.

\subsection{Kompatibilität}
Auch sei nicht unerwähnt, dass zwar eine grundsätzliche Kompatibilität zu Microsoft-Dateiformaten gegeben ist, es meist dennoch zu Problemen kommt, möchte man beispielsweise ein MS-Word Dokument oder eine Excel-Tabelle mit LibreOffice öffnen. Meist sind es Aspekte der Formatierung, aber auch Funktionsnamen in Excel-Tabellen, die sich unterscheiden können.