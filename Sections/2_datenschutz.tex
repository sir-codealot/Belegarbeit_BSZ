Datenschutz ist sowohl im Privat- als auch im Arbeitsumfeld wichtig. Dabei können die beiden Bereiche für den Einzelnen mitunter schwer abgegrenzt werden. Denn zum einen verarbeitet man gegebenenfalls selbst personenbezogene Daten Dritter, zum anderen werden auch die eigenen Informationen permanent bei der Nutzung verschiedenster Programme verarbeitet. Dabei kommt es immer wieder vor, dass die Daten nicht datenschutzkonform verarbeitet werden oder dass das dafür zuständige Personal nicht ausreichend sensibilisiert oder geschult ist.\\

Unter dem Begriff Datenschutz versteht man den Schutz der eigenen - personenbezogenen - Daten und das grundlegende Recht über die Preisgabe und Verarbeitung dieser Daten selbst zu bestimmen. In Deutschland wird dies durch das Grundrecht auf informationelle Selbstbestimmung\footnote{gestützt auf Artikel 1 \& 2 Grundgesetz} festgeschrieben. Das bedeutet konkret, dass jede natürliche Person alleinig über die Erhebung sie betreffender personenbezogener Daten zustimmen oder auch widersprechen kann. Ein Widerspruch kann zu jeder Zeit, auch nach vorangegangener Erlaubniserteilung, erfolgen. Somit hat jeder natürliche Person zu jeder Zeit die uneingeschränkte Handhabe über die sie betreffenden Daten.

\subsection{Personenbezogene Daten}
    Unter personenbezogenen Daten versteht man sämtliche Daten, die einer durch sie identifizierten oder identifizierbarbaren Person zugeordnet sind. Darunter zählen alltägliche Daten wie der eigene Name, die eigene Adresse, das Geburtsdatum, Kennnummern wie die Personalausweis -, Sozialversicherungs - oder Kontonummer, Online-Kennungen wie Login- und Nicknames,\\
    Standortdaten und IP-Adressen oder die eigene Telefonnummer.\\
    Aber auch Daten, die über den physischen, physiologischen oder psychischen Zustand, die eigene Krankheitshistorie, sexuelle Orientierung oder die politische Einstellung Aufschluss geben, fallen ebenfalls unter diesen Sammelbegriff, werden aber aufgrund ihrer besonderen Sensibilität als \glqq besondere personenbezogene Daten\grqq{} speziell behandelt. Ihrer Verarbeitung muss stets separat zugestimmt werden.

\subsection{Europäische Rechtslage}
    In Europa gilt seit Mai 2018 in Form der \glqq Datenschutzgrundverordnung\grqq{} eine für alle Mitgliedstaaten der Europäischen Union bzw. Staaten des europäischen Wirtschaftsraumes verbindliche Verordnung, die den Datenverkehr innerhalb dieser regelt. Diese wird durch die jeweilige nationale Gesetzgebung der einzelnen Länder ergänzt, in Deutschland in Form des Bundesdatenschutzgesetzes.


\subsection{Vergleich von US- und EU-Datenschutz}
    Hier sollen die Unterschiede nach beiden Rechtslagen aufgezeigt werden.

\subsection{IST-Zustand}
    Hier wird der Stand der Dinge erläutert.

\subsection{Probleme mit US-Software und deren Nutzung in Unternehmen}
    Der Kern der Belegarbeit. Hier wird abgehandelt, inwiefern Software von US-Unternehmen den europäischen Datenschutzstandards entspricht, wo es Reibungspunkte gibt und wo klare Verstöße vorliegen.

\subsection{Bekannte Verstöße}
    Hier werden ein paar bekannte Verstöße aufgezeigt, z. B. von Whatsapp, Facebook, Microsoft (dort speziell Windows und MS Office Suites).

\subsection{Freie Software-Alternativen}
    Hier soll auf alternativefreie Software aufmerksam gemacht werden und bereits durchgeführte oder in der Durchführung befindliche Projekte (LiMux, Schleswig-Holstein) und deren Erfolg analysiert werden.