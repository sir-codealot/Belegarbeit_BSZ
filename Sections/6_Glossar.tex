\renewcommand*{\arraystretch}{1.4}
\begin{longtable}{p{0.25\textwidth}p{0.75\textwidth}}
    ICMP & Internet Control Message Protocol - ein Protokoll mit dem überprüft werden kann, ob ein Host im Netzwerk aktiv ist
    \\
    TTL & time to live, gibt beim ICMP die verbleibende maximale Lebenszeit im Netzwerk in Sekunden an
    \\
    MTU & Maximum Transmission Unit, maximale Größe unfragmentierter Datenpakete
    \\
    loopback & Schleifenschaltung mit Nachrichten- oder Informationskanal in dem Sender und Empfänger identisch sind
    \\
    ping & ein Konsolenbefehl, welcher unter fast allen Betriebssystemen funktioniert und ICMP Protokolls Pakete zu interfaces sendet und zurückbekommt, ob diese aktiv sind oder nicht
    \\
    interface &  zu deutsch “Schnittstelle”, hier in dieser Dokumentation wird zumeist mit interface eine virtuelle oder physische Schnittstelle zwischen Netzen gemeint
    \\
    header & Bei Rechnernetzwerken besitzt jedes von einem Rechner versandte Datenpaket einen Header, der Daten über den Absender, Empfänger, Typ und Lebensdauer des Datenpakets enthält.
    \\
    & Beim Hypertext Transfer Protocol (HTTP) werden über den Header HTTP-Cookies und Informationen wie Dateigröße übertragen
    \\
    Payload & Nutzdaten, die keine Steuer- oder Protokollinformationen enthalten. Nutzdaten sind unter anderem Sprache, Text, Zeichen, Bilder und Töne.
    \\
    OSI Schichtmodell & Modell, welches die Ebenen die ein Netzwerk ausmachen beschreibt
    \\
    DoD Schichtmodell & Modell welches Datenübertragungen darstellt
    \\
    link-layer adress & feste Adressen wie die MAC Adresse
\end{longtable}