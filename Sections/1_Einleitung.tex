Täglich werden von Unternehmen aus allen Branchen Daten unterschiedlichster Art verarbeitet, von Kundenkorrespondenz über das Verarbeiten personenbezogener Daten im Sinne geltendem eurpäischen Rechtes bis hin zur Auswertung interner Produktionsabläufe und anderer Schaffensprozesse. Sei es die E-Mail an den Kollegen, die Messenger-Nachricht im Gruppenchat oder das als Videokonferenz abgehaltene Team-Meeting.\\
Dabei kommen zumeist Betriebssysteme, Büro- und Branchensoftware, aber auch Software und Apps die jeder auch aus dem privaten Umfeld kennt, von führenden Herstellern mit Sitz der Firmenzentrale meist außerhalb des europäischen Wirtschaftsraumes zum Einsatz. Namenhafte Vertreter sind hier beispielsweise Microsoft, Google, Zoom oder Facebook.\\
Nun sind die Hersteller zwar verpflichtet, in Europa vertriebene Produkte den hier gesetzlich geregelten Datenschutzbestimmungen anzupassen und müssen diesen auch verbindlich umsetzen. Das dies nicht immer gelingt, weil die jeweiligen Produkte nur teilweise oder aber auch überhaupt nicht den rechtlichen Gegebenheiten angepasst werden oder diese Anpassung seitens der Hersteller sehr viel Zeit in Anspruch nimmt, soll Thema dieser Belegarbeit sein.\\
Nun ist es auch so, dass die gesamte Thematik sehr komplex ist und in ihrer Gänze den Rahmen dieser Arbeit sprengen würde. Im Hauptteil wird deshalb für das bessere Verständnis nur soweit auf den Punkt des Datenschutzes eingegangen, dass die angeführten Beispiele und Probleme gut erfasst werden können. 
