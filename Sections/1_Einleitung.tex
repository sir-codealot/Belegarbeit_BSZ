Täglich werden von Unternehmen aus allen Branchen Daten unterschiedlichster Art erhoben, von Kundenkorrespondenz über das Verarbeiten personenbezogener Daten im Sinne geltendem europäischen Rechtes bis hin zur Auswertung interner Produktionsabläufe und anderer Schaffensprozesse. Sei es die E-Mail an den Kollegen, die Messenger-Nachricht im Gruppenchat oder das als Videokonferenz abgehaltene Team-Meeting.\\
Dabei kommen im Büroalltag zumeist Betriebssysteme und Büro-Standardsoftware von Microsoft zum Einsatz, aber auch Software und Apps die jeder aus seinem privaten Umfeld kennt, von führenden Herstellern mit Sitz der Firmenzentrale meist außerhalb des europäischen Wirtschaftsraumes wird verwendet. Namhafte Vertreter sind hier beispielsweise Microsoft, Google, Zoom oder Facebook.\\
Nun sind die Hersteller zwar verpflichtet, in Europa vertriebene Produkte den hier gesetzlich geregelten Datenschutzbestimmungen anzupassen und müssen diese auch verbindlich umsetzen. Dass dies nicht immer auf Anhieb gelingt, weil die jeweiligen Produkte nur teilweise oder gänzlich nicht den rechtlichen Gegebenheiten angepasst werden oder diese Anpassung seitens der Hersteller sehr viel Zeit in Anspruch nimmt, soll Thema dieser Belegarbeit sein.\\
Diese Arbeit beleuchtet den europäischen und US-amerikanischen Datenschutz und stellt beide gegenüber, um die bestehende Diskrepanz zwischen den beiden Gesetzgebungen zu\\
erläutern. Danach wird der Fokus auf Produkte von Microsoft für den Büroalltag gelegt, im speziellen Windows 10 und Microsoft Office und wie und in welchem Umfang durch diese Produkte personenbezogene Daten erhoben werden. In diesem Zusammenhang soll auch geklärt werden, ob die Erhebung immer mit den gesetzlichen Gegebenheiten konform sind.