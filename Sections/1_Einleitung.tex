Täglich werden von Unternehmen aus allen Branchen Daten unterschiedlichster Art verarbeitet, von Kundenkorrespondenz über das Verarbeiten personenbezogener Daten im Sinne geltendes eurpäischem Rechtes bis hin zur Auswertung interner Produktionsabläufe und anderer Schaffensprozesse. Seien es E-Mails unter Kollegen, die Nachricht im Messenger oder das als Videokonferenz abgehaltene Team-Meeting.\\
Dabei kommen zumeist Betriebssysteme, Büro- und Branchensoftware führender Hersteller zum Einsatz, die ihren Hauptsitz meist außerhalb des europäischen Wirtschaftsraumes haben. Namenhafte Vertreter sind hier beispielsweise Microsoft, Zoom oder Facebook.\\
Nun sind die Hersteller zwar verpflichtet, in Europa vertriebene Produkte den hier gesetzlich geregelten Datenschutzbestimmungen anzupassen und müssen diesen auch verbindlich umsetzen. Das dies nicht immer gelingt, weil die jeweiligen Produkte nur teilweise oder aber auch überhaupt nicht datenschutzkonform umgesetzt werden, soll Thema dieser Belegarbeit sein.
