Im Folgenden wird das Projekt \glqq Contact Search Tool\grqq{} dokumentiert. Dieses Projekt wurde im Rahmen der Abschlussprüfung zum Fachinformatiker für Anwendungsentwicklung\footnote{äquivalent Übungsprojekt} erstellt. Der Ausbildungsbetrieb ist Unitedprint.com SE in Radebeul. Unitedprint zählt zu den großen E-Commerce-Unternehmen im Bereich Druck in Europa. Das Unternehmen besitzt eine eigene IT-Abteilung, welche sich unter anderem mit der Entwicklung neuer Features und der Wartung der eigenen Webportale sowie der Pflege der unternehmenseigenen Administration befasst. Dieses Projekt ist eines der zahlreichen internen Projekte, in denen das Unternehmen gleichzeitig auch der Auftraggeber ist.\\

\subsection{Auftraggeber}
    Auftraggeber und zugleich Kunde des Projektes ist das Backend Relaunch Team, welches Teil der Abteilung IT-Development des ausbildenden Unternehmens ist. Dieses befasst sich mit dem Neubau der unternehmenseigenen Administration, eine Sammlung verschiedener für den internen Betrieb benötigter Tools, welche den jeweiligen Abteilungen für die Bearbeitung der unterschiedllichen Abläufe zur Verfügung stehen.\\
    Das Team steht während der Entwicklung und Testphase des Projekt-Tools als Ansprechpartner für den Auszubildenden zur Verfügung.

\subsection{Problembeschreibung}
    Die momentan genutzte Unternehmens-Administration ist technisch überholt und über ihre Nutzungszeit hinweg immer weiter durch neu hinzugekommene Werkzeuge  gewachsen, die zwischenzeitlich teils kaum oder gar keine Verwendung finden, sich aber trotzdem noch im Codebestand befinden und bestenfalls lediglich im Frontend deaktiviert wurden.

\subsection{Projektbeschreibung}
    Im Zuge des Relaunch der Unternehmens-Administration soll der Bestand bereitgestellter Tools überarbeitet und die Neuimplementierung dieser mit angepassten Anforderungen vorgenommen werden. Eines der benötigten Tools ist das Kontakt-Tool, über welches sich Mitarbeiterdaten abrufen lassen. Dieses bietet, neben den für alle Mitarbeiter einsehbaren Daten, der Abteilung Human Resources die Möglichkeit, eine umfangreichere Ansicht der sogenannten Stammdaten aufzurufen.

\vfill
\pagebreak

\subsection{Projektbegründung}
    Da es zum einen immer schwieriger wird, Perl-Entwickler zu finden und zum anderen Perl technisch von anderen verfügbaren Sprachen in seiner Leistungsfähigkeit überholt wurde, hat sich der Auftraggeber dazu entschieden, die Admin auf PHP- und Next.js-Basis von Grund auf neu zu implementieren. Damit geht einher, dass der Bestand an verfügbaren Tools geprüft und überarbeitet wird und bei Tools, die übernommen werden, die jeweiligen Anforderungen neu bewertet werden und Funktionalitäten entfallen oder hinzugefügt werden müssen.\\
    Das Kontakt-Tool ist eines der Tools, welches neu und ob dessen abteilungsübergreifender Wichtigkeit für den reibungslosen betrieblichen Ablauf auch als eines der Ersten implementiert werden soll.

\subsection{Projektziele}
    Mit Abschluss des Projektes soll ein funktionierendes Tool zum Suchen von aktiven Mitarbeitern und dem Anzeigen zugehöriger Daten entstehen. Die neue Anwendung soll die gleiche Funktionalität bezüglich der Suchfunktion wie die derzeit Genutzte aufweisen, jedoch in technisch optimierter Form im Relaunch der Administration implementiert werden. Des Weiteren soll der Umfang der angezeigten Daten reduziert werden, da viele bisherig angezeigte Informationen nicht mehr mit aktuellen Datenschutzbestimmungen vereinbar sind.\\
    Das Tool wird jedoch zum Zeitpunkt der Abgabe noch nicht produktiv genutzt werden, da im Relaunch derzeitig noch keine Rechteverwaltung implementiert ist und es dadurch nicht authorisierten Nutzern einen zu großen Einblick in die Stammdaten der Mitarbeiter gewährt.

\vfill
\pagebreak