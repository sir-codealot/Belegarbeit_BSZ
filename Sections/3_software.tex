\subsection{Platzhirsch Microsoft}
Schauen wir uns nun die Umsetzung des Datenschutzes in Software von Betriebssystemen und Bürosoftware von Microsoft an und ob dieser immer auch im Sinne des EU-Rechtes umgesetzt wird beziehungsweise wo es Reibungspunkte gibt.

\subsection{Endnutzer-Lizenzvertrag}
Der sogenannte Endnutzer-Lizenzvertrag (\glqq End User Licence Agreement\grqq{}, kurz EULA) stellt wie es der Name bereits sagt den Nutzungsvertrag für eine Software dar und sollte jeder Software beiliegen, welche dies durch ihr zugrunde liegendes Geschäftsmodell erfordert. Dabei spielt es keine Rolle, ob es sich um bezahlte oder kostenlose Programme handelt. In den \mbox{EULAs} wird unter anderem festgelegt, in welchem Umfang personenbezogene Daten gesammelt, ausgewertet, und gegebenenfalls auch mit Dritten geteilt werden. Bei Konzernen mit mehreren Tochterunternehmen wie zum Beispiel der Facebook Inc. würde eine Weitergabe dieser Daten an Dritte bedeuten, dass die durch die Nutzung von beispielsweise Instagram erhobenen Daten auch an andere unternehmenseigene Dienste wie Facebook oder Whatsapp übermittelt werden dürfen und somit effizienter verarbeitet werden können. Aber auch ein Weiterverkauf oder eine unentgeltliche Weitergabe an andere Unternehmen ist im Bereich des Möglichen.\\
Diese Praktiken sind nach europäischen Recht prinzipiell untersagt, weswegen es für Europa meist angepasste EULAs für die jeweiligen Produkte und Dienste gibt, in denen diese Aspetke entweder nicht berücksichtigt oder ausgeschlossen werden.

\subsection{Situation im Büroalltag}
Blickt man in die Büros von Unternehmen, Ämtern und Regierungsinstitutionen, wird man wenig Varianz bei der genutzten Software finden. In der Regel kommen Software-Lösungen von Microsoft in Form von Betriebssystemen und Bürosoftware zum Einsatz, was sie durch ihre hohe Verbreitung zur Standardsoftware erhebt. Weiter wird durch diesen Umstand eine völlige Abkehr von Microsoft-Produkten erschwert, weil wiederum Hersteller von Spezialsoftware ihre Produkte nur für Windows-Betriebssysteme anbieten oder nur Schnittstellen für Microsoft Office bereitstellen. Somit kristallisiert sich ein essenzielles Problem heraus: die Abhängigkeit gegenüber eines Herstellers.


\subsection{Erhebung durch Windows 10}
Microsoft-Software, wie auch Produkte anderer Hersteller, sammeln sogenannte Telemetriedaten. Unternehmen haben an diesen Daten prinzipiell ein berechtigtes Interesse, da sie Aufschluss über das Nutzerverhalten geben und die Hersteller dadurch Fehlverhalten in der Software erkennen oder ob eingebaute Features genutzt werden bzw. wo Verbesserungspotenzial besteht. Die Daten werden zwar anonymisiert übermittelt, die Sammelwut gerade bei Windows 10, vor allem in den ersten Jahren seit Einführung, ging oftmals weit über das benötigte Maß hinaus. Womit ein grundsätzlicher Verstoß gegen das wesentliche Prinzip der Datensparsamkeit der DSGVO vorlag. Bei einem frisch aufgesetzten Windows 10 sind Dienste standardmäßig aktiviert, die beispielsweise den Standort abfragen, Stimmen aufzeichnen und auswerten, damit verbunden sind Mikrofon und Kamera von Haus aus systemweit und für jedes Programm aktiviert. Das Benachrichtigungssystem lässt es zu, dass Programme Benachrichtigungen senden und teils sogar mitlesen dürfen, welche von dritten Programmen gesendet werden (beispielsweise eine Benachrichtigung über eine E-Mail mit einem Termin, welcher direkt aus dem Benachrichtigungssystem von der Kalender-Anwendung abgegriffen wird) oder auch der Zugriff auf das Telefonbuch eines verknüpften Telefons. Für alle diese Dienste muss der Nutzer bei der Ersteinrichtung zum Deaktivieren aktiv werden (sogenanntes Opt-out). Wird die Einrichtung durch einen Systemadministrator vorgenommen, kann der Nutzer die Einstellungen zwar immer noch ändern - zu finden in den Systemeinstellungen im Bereich \glqq Datenschutz\grqq{} - allerdings kommt hinzu, dass viele Nutzer sich gar nicht über die im Hintergrund laufenden Dienste bewusst sind. Für eine datenschutzorientierte Konfiguration von Windows 10 stellt die deutsche Verbraucherschutzzentrale einen umfassenden Leitfaden bereit.\footnote{von verbraucherzentrale.de (06.01.2020), Datenschutz bei Windows 10 erhöhen, aufgerufen am 27.03.2021, https://www.verbraucherzentrale.de/wissen/digitale-welt/datenschutz/datenschutz-bei-windows-10-erhoehen-12154} Eine komplette Deaktivierung der Datenübermittlung ist momentan nur über Umwege über die Gruppenrichtlinien von Windows (also nicht direkt über die Systemeinstellungen) und auch nur in den Enterprise- und Education-Editionen möglich.\footnote{Moritz Tremmel (3.02.2020), Windows 10 lässt sich ohne Telemetrie betreiben, aufgerufen am 31.03.2021, https://www.golem.de/news/datenschutzbeauftragter-windows-10-laesst-sich-ohne-telemetrie-betreiben-2002-146423.html} In den Editionen Home und Pro hat man nur die Möglichkeit, über die Systemeinstellung im Untermenü \textit{Datenschutz $\rightarrow$ Diagnose und Feedback} die Übermittlung auf \glqq erforderliche Diagnosedateien\grqq{} zu stellen und somit lediglich auf ein Minimum zu reduzieren.

Dies resultiert ebenfalls durch die europäische Gesetzgebung, da die Erhebung im privaten Raum durchaus zulässig ist, weil jeder Einzelne den EULAs individuell zustimmt und somit auch dem Verwerten der eigenen Daten. Bei Unternehmen und öffentlichen Stellen ist dies nicht umsetzbar, weswegen dort restriktivere Gesetze gelten. Somit tut Microsoft mehr oder weniger das Nötigste, dass Windows-Systeme weiterhin rechtskonform im öffentlichen Raum nutzbar bleiben.

Microsofts Datenschutzbestimmungen sind im Übrigen auf allen Unterseiten im Menü \glqq Datenschutz\grqq{} verlinkt. Das Dokument ist sehr umfangreich, kleinteilig und außerordentlich offen beschrieben und vermittelt beim Querlesen den Eindruck, dass alle erdenklich sammelbaren Daten auch gesammelt werden\footnote{Hajo Schulz (24.08.2018), Privatsphäre in Windows 10 schützen, aufgerufen am 31.03.2021, https://www.heise.de/ct/artikel/Privatsphaere-in-Windows-10-schuetzen-4140586.html}. Immerhin stellt Windows die Möglichkeit bereit, Protokolldateien von übermittelten Daten zu speichern und mittels dem hauseigenen Programm \glqq Diagnostic Data Viewer\grqq{} zu sichten.

\subsection{Microsoft Office und Office 365}
Auch durch Microsoft Office werden Daten erhoben, die durch die Benutzung der Produkte entstehen. Diese werden über den Windows-Dienst \glqq Office-Telemetrie-Prozessor\grqq{} gesammelt, in einer Datenbank abgelegt und in einer Übersicht für Systemadministratoren aufbereitet - das sogenannte \glqq Office-Telemetrie-Dashboard\grqq{}.\\
Das Sammeln der Nutzerdaten ist auch hier grundsätzlich erst einmal nichts Verwerfliches. Auch Produkte anderer Hersteller, beispielsweise LibreOffice, bitten darum Nutzerdaten sammeln zu dürfen, bieten aber auch die Möglichkeit diese Funktionalität komplett zu deaktivieren. Mit der \glqq Office 365\grqq{}-Suite bietet Microsoft ein erheblich umfangreicheres Produkt als die Offline-Office-Varianten (z. B. Microsoft Office Professional Plus 2019) an, die einen administrativen Vorteil gegenüber diesen haben. Die cloudbasierte Office-Variante bietet als eine Art \glqq Rolling Release\grqq{} stets aktuelle Features und Fehlerbeseitungen, ohne dass Unternehmen gezwungen sind, auf ihren Arbeitsplätzen eine komplett neue Office-Version installieren zu müssen, um diese Funktionen nutzen zu können. Jedoch sammelt Office 365 auch wesentlich akribischer Nutzerdaten, als es bei seinem Offline-Pendant der Fall ist.

\subsection{My Analytics und personalisierter Score}
Die Datenauswertung, speziell wenn man sich das Microsoft Office 365 Feature \glqq My Analytics\grqq{} näher anschaut, wirkt hingegen mindestens besorgniserregend. Die Ende 2020 freigegebene und optional nutzbare Funktion analysiert die Ressourcenauslastung der einzelnen Mitarbeiter und bereitet diese auf. Die damit einhergehende \glqq Workspace Analytics\grqq{} genannte Übersicht errechnet über Benchmarks den sogenannten \glqq Productivity Score\grqq{}, ein Wert, der die Arbeitsleistung visualisieren soll. Dabei werden auch Hardware-Daten gesammelt und ausgewertet, beispielsweise über die Startzeit des PCs. Microsoft nennt dies \glqq Technology Experience\grqq{}.

Der personalisierte Score gibt einen sehr detailierten Einblick in die Produkivität und Arbeitsweise der einzelnen Mitarbeiter. So werden Dinge wie die Anzahl der versendeten E-Mails oder Yammer-Beiträge (Microsofts soziales Netzwerk für Unternehmen), in OneDrive benutzte oder darüber weitergeleitete hochgeladene Dateien, über Microsoft Teams versendete Chat-Nachrichten und gehaltene Videokonferenzen, die Teilnahmezeit an Videokonferenzen, die Dauer des Screen Sharings durch einen Mitarbeiter oder auch dessen \glqq Fokus-Zeit\grqq{}, also die Zeit, in der er keine der ausgewerteten Funktionalitäten verwendet hat, ausgewertet. Das Ganze passierte bei Einführung des Features mit der eindeutigen Zuordnung des jeweiligen Nutzernamens, Gruppenzugehörigkeiten innerhalb des Unternehmens-Office-Ökosystems und dem Standort zu den erhobenen Daten. Eine Anonymisierung war zwar möglich, die Einrichtung aber optional. Hier hat Microsoft in der Zwischenzeit auf Drängen von Aktivisten wie dem Österreicher Wolfie Christl und anderen nachgebessert. Zumindest werden keine Nutzerkennungen mehr bei den Auswertungen angezeigt, wohl aber in drei Kategorien die Gerätenummern, was ebenfalls Rückschlüsse auf den Mitarbeiter zulässt.

All diese Daten sind durch die Admins in den Unternehmen frei einsehbar, sodass diese, wenn man es wohlwollend betrachtet, Verbesserungstipps in Bezug auf die Arbeitsweise und Best Practice an die Mitarbeiter kommunizieren können. Jedoch liegt hier auch der größte Kritikpunkt an My Analytics. Denn umgekehrt lassen sich durch die eindeutige Zuordnung auch vermeintliche Rückschlüsse über die Produktivität jedes Mitarbeiters anstellen, der durch den mehr oder weniger großen Interpretationsspielraum auch angreifbar wird.

Die erhobenen Daten sind allerdings nur dann wirklich aussagekräftig, wenn für die gesamte Unternehmenskommunikation auf Microsoft-Produkte gesetzt wird. Sobald nicht Microsoft Teams oder Skype für Chat und Videokonferenzen genutzt wird, sondern ein anderes nicht von Microsoft stammendes Programm, entfällt dies folglich in der Statistik und verfälscht den Score. Somit ist auch die spätere Auswertung und der errechnete Score nicht mehr verwertbar.\\

Gegenüber der Fachzeitschrift \textit{c't} erklärte der Rechtsexperte des Deutschen Gewerkschaftsbundes Bertold Brüchner: \glqq Funktionen, mit denen Unternehmen die Arbeitsgepflogenheiten ihrer Bürobelegschaft detailliert durchleuchten können, widersprechen und verstoßen gegen Persönlichkeitsrechte der Mitarbeiter, Datenschutz und – wenn vorhanden – den Beteiligungsrechten- und -pflichten der Betriebs- oder Personalräte.\grqq{}\footnote{Dr. Hans-Peter Schüler: \glqq Microsoft 365, Zeitgeist 1984\grqq{}, c't Ausgabe 25/2020 S.40} Laut Brüchners Einschätzung ist ein rechtskonformer Einsatz des Features ausgeschlossen, da dies einer Überwachung am Arbeitsplatz gleich kommt und somit gegen geltendes Persönlichkeitsrecht\footnote{Grundlage: Art. 1 Abs. 1 GG (Menschenwürde), Art. 2 Abs. 1 GG (Freie Entfaltung der Persönlichkeit)} verstößt.\\

\vfill