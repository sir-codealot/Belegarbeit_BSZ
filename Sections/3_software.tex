\section{Software}
Im zweiten Teil dieser Arbeit wird auf die Umsetzung des Datenschutzes in Software von US-Unternehmen eingegangen und ob dieser immer auch im Sinne des EU-Rechtes umgesetzt wird beziehungsweise wo es Reibungspunkte gibt. Hierbei wird im Besonderen das Augenmerk auf Produkte von Microsoft oder Google, aber auch Programme anderer Hersteller wie Whatsapp als Messenger oder auch Video-Konferenz Programme wie Zoom, welche momentan durch die Pandemie stark an Relevanz gewonnen haben.

\subsection{Probleme mit US-Software und deren Nutzung in Unternehmen}
Blickt man in die Büros deutscher Unternehmen, Ämter und Regierungsinstitutionen wird man wenig Varianz in der benutzten Software finden. In der Regel kommen Software-Lösungen Von Microsoft in Sachen Betriebssysteme und Bürosoftware zum Einsatz, was sie durch ihre hohe Verbreitung zur Standardsoftware erhebt. Weiter wird durch diesen Umstand eine völlige Abkehr von Microsoft-Produkten erschwert, weil wiederum Hersteller von Spezialsoftware ihre Produkte nur für Windows Betriebssysteme anbieten oder nur Schnittstellen für Microsoft Office bereitstellen.\\
Somit kristallisiert sich bereits das erste Problem heraus: die allgemeine Abhängigkeit gegenüber einem Herstellers. \\
Die weite Verbreitung 

\subsection{Bekannte Verstöße}
Hier werden ein paar bekannte Verstöße aufgezeigt, z. B. von Whatsapp, Facebook, Microsoft (dort speziell Windows und MS Office Suites).

\subsection{Freie Software-Alternativen}
Hier soll auf alternativefreie Software aufmerksam gemacht werden und bereits durchgeführte oder in der Durchführung befindliche Projekte (LiMux, Schleswig-Holstein) und deren Erfolg analysiert werden.