\section{Endnutzer-Lizenzverträge}
Die sogenannten Endnutzer-Lizenzverträge (\glqq End User Licence Agreement\grqq{}, EULA) stellen wie es der Name sagt den Nutzungsvertrag für eine Software dar. In Ihnen wird unter anderem auch festgelegt, in welchem Umfang personenbezogene Daten gesammelt, ausgewert, und gegebenenfalls auch mit Dritten geteilt werden. Bei Konlgomeraten mit mehreren Tochterunternehmen wie zum Beispiel der Facebook Inc. würde eine Weitergabe dieser Daten an Dritte bedeuten, dass die durch die Nutzung von Instagram erhobenen Daten auch an andere Dienste wie beispielsweise Facebook oder Whatsapp übermittelt werden dürfen und somit effizienter verarbeitet werden können. Aber auch ein Weiterverkauf an andere Unternehmen ist im Bereich des Möglichen.\\
Diese Praktiken sind nach europäischen Recht prinzipiell untersagt, weswegen es für Europa meist angepasste EULAs für die jeweiligen Produkte und Dienste gibt, in denen diese Aspetke nicht berücksichtigt sind.

\section{Office-Software von Microsoft}
Im zweiten Teil dieser Arbeit wird auf die Umsetzung des Datenschutzes in Software von US-Unternehmen eingegangen und ob dieser immer auch im Sinne des EU-Rechtes umgesetzt wird beziehungsweise wo es Reibungspunkte gibt. Hierbei wird im Besonderen das Augenmerk auf Produkte von Microsoft oder Google, aber auch Programme anderer Hersteller wie Whatsapp als Messenger oder auch Video-Konferenz Programme wie Zoom, welche momentan durch die Pandemie stark an Relevanz gewonnen haben.

\subsection{Situation im Arbeitsalltag und auftretende Probleme}
Blickt man in die Büros der Unternehmen, Ämter und Regierungsinstitutionen, wird man wenig Varianz bei der genutzten Software finden. In der Regel kommen Software-Lösungen Von Microsoft in Sachen Betriebssysteme und Bürosoftware zum Einsatz, was sie durch ihre hohe Verbreitung zur Standardsoftware erhebt. Weiter wird durch diesen Umstand eine völlige Abkehr von Microsoft-Produkten erschwert, weil wiederum Hersteller von Spezialsoftware ihre Produkte nur für Windows Betriebssysteme anbieten oder nur Schnittstellen für Microsoft Office bereitstellen. Somit kristallisiert sich bereits das erste Problem heraus: die allgemeine Abhängigkeit gegenüber einem Herstellers.


\subsection{Windows 10}
Hinzu kommt, dass Microsoft-Produkte, wie auch Produkte anderer Hersteller, sogenannte Telemetriedaten erheben, die Aufschluss über das Nutzerverhalten geben. Diese Daten werden zwar anonymisiert übermittelt, die Sammelwut gerade bei Windows 10 und in den ersten Jahren der Einführung ging oftmals weit über das benötigte Maß hinaus. Womit grundsätzlich schon ein Verstoß ein wesentliches Prinzip der DSGVO stattgefunden hat: das Prinzip der Datensparsamkeit. Bei einem frisch aufgesetzten Windows 10 System sind Dienste standardmäßig aktiviert, die beispielsweise den Standort abfragen, Stimmen aufzeichnen und auswerten, damit verbunden sind Mikrofon und Kamera von Haus aus systemweit und somit für jedes Programm aktiviert, das Benachrichtigungssystem lässt es ebenfalls zu, dass Programme Benachrichtigungen senden und teils sogar mitlesen dürfen,
welche von dritten Programmen gesendet werden (beispielsweise eine Benachrichtigung über eine Email mit digitaler Visitenkarte, auf die wiederum die Kontakte-Anwendung direkten Zugriff hat und diese Daten weiterverarbeiten kann) oder auch der Zugriff auf das Telefonbuch eines verknüpften Telefons. Für alle diese Dienste muss der Nutzer bei bei der Ersteinrichtung zum Deaktivieren aktiv werden (sogenanntes Opt-Out). Wird diese durch einen Systemadministrator vorgenommen kann der Nutzer die Einstellungen allerdings immernoch ändern - zu finden in den Systemeinstellungen im Bereich \glqq Datenschutz\grqq{}. Allerdings kommt auch hinzu, dass viele Nutzer sich gar nicht über die im Hintergrund laufenden Dienste bewusst sind. Für eine datenschutzorientierte Konfiguration von Windows 10 stellt die deutsche Verbraucherschutzzentrale einen umfassenden Leitfaden bereit\footnote{https://www.verbraucherzentrale.de/wissen/digitale-welt/datenschutz/datenschutz-bei-windows-10-erhoehen-12154}. Eine komplette Deaktivierung der Datenübermittlung ist des Weiteren trotzdem nicht möglich. Man bekommt nur die Möglichkeit über die Systemeinstellung im Untermenü \textit{Datenschutz $\rightarrow$ Diagnose und Feedback} die Übermittlung auf \glqq erforderliche Diagnosedateien\grqq{} zu stellen und somit lediglich auf ein Minimum zu reduzieren.\\
Microsofts Datenschutzbestimmungen sind im Übrigen auf allen Unterseiten im Menü \glqq Datenschutz\grqq{} verlinkt. Das Dokument ist sehr umfangreich, kleinteilig und außerordentlich offen beschrieben und vermittelt beim Querlesen den Eindruck, dass alle erdenklich sammelbaren Daten auch gesammelt werden. Immerhin stellt Windows die Möglichkeit bereit, Protokolldateien von übermittelten Daten zu speichern und mittels dem hauseigenen Programm \glqq Diagnostic Data Viewer\grqq{} zu sichten.

\subsection{Microsoft Office und Office 365}
Auch über Microsoft Office werden Daten gesammelt, die durch die Benutzung der Produkte entstehen. Diese werden über den Windows-Dienst \glqq Office-Telemetrie-Prozessor\grqq{} gesammelt, in einer Datenbank abgelegt und in einer Übersicht für Systemadministratoren aufbereitet - das sogenannte \glqq Office-Telemetrie-Dashboard\grqq{}.\\
Das Sammeln der Nutzerdaten ist auch hier im Grunde erst einmal nichts Verwerfliches. Auch Produkte anderer Hersteller, beispielsweise LibreOffice, bitten darum Nutzerdaten sammeln zu dürfen, bieten aber auch die Möglichkeit diese Funktionalität komplett zu deaktivieren\footnote{näheres zu den Alternativen im Abschnitt \glqq Freie Software-Alternativen\grqq{}}. Microsoft Office 365 bietet Unternehmen einen erheblichen administrativen Vorteil gegenüber den herkömmlichen Office-Produkten (z. B. Microsoft Office Professional Plus 2019). Die cloudbasierte Office-Suite bietet als eine Art \glqq Rolling Release\grqq{} stets aktuelle Features und Fehlerbeseitungen, ohne dass Unternehmen gezwungen sind, auf ihren Arbeitsplätzen eine komplett neue Office-Version installieren zu müssen, um diese Funktionen nutzen zu können. Jedoch sammelt Office 365 auch erheblich akribischer Daten der Nutzer, als es sein \glqq Offline\grqq{}-Pendant tut.

\subsection{My Analytics und personalisierter Score}
Die Datenauswertung, speziell wenn man sich das Microsoft Office 365 Feature \glqq My Analytics\grqq{} näher anschaut, wirkt hingegen mindestens besorgniserregend. Die Ende 2020 freigegebene und optional nutzbare Funktion analysiert die Ressourcenauslastung der einzelnen Mitarbeiter und bereitet diese auf. Die damit einhergehende \glqq Workspace Analytics\grqq{} Übersicht errechnet über Benchmarks\footnote{kontinuierliche Vergleichsanalyse von bestimmten Kennzahlen} den sogenannten \glqq Productivity Score\grqq{}, einen Wert, der die Arbeitsleistung visualisieren soll. Dabei werden auch Hardware-Daten gesammelt und ausgewertet, beispielsweise über die Startzeiten PCs. Microsoft nennt das \glqq Technology Experience\grqq{}.\\

Der personalisierte Score gibt einen sehr detailiierten Einblick in die Produkivität und Arbeitsweise der einzelnen Mitarbeiter. So werden Dinge wie die Anzahl der versendeten E-Mails oder Yammer-Beiträge\footnote{Microsofts soziales Netzwerk für Unternehmen}, Chat-Nachrichten und Videokonferenzen über Microsoft Teams, in OneDrive benutzten oder  darüber weitergeleiteten hochgeladenen Dateien, die Teilnahmezeit an Videokonferenzen und wie lange der Nutzer seinen Bildschirm geteilt hat oder auch die \glqq Fokus-Zeit\grqq{} des Mitarbeiters, also die Zeit, in der er keine der ausgewerteten Funktionalitäten verwendet hat. Das Ganze passierte bei Einführung des Features mit der eindeutigen Zuordnung des jeweiligen Nutzernamens, Gruppenzugehörigkeiten innerhalb des Unternehmens-Office-Ökosystems und dem Standort. Eine Anonymisierung war zwar möglich, die Einrichtung aber optional. Hier hat Microsoft in der Zwischenzeit auf Drängen von Aktivisten wie dem Österreicher Wolfie Christl und anderen nachgebessert. Zumindest werden keine Nutzerkennungen mehr bei den Auswertungen angezeigt, wohl aber in drei Kathegorien die Gerätenummern, was ebenfalls Rückschlüsse auf den Mitarbeiter zulässt.\\

All diese Daten sind durch die Admins in den Unternehmen frei einsehbar, sodass diese, wenn man es wohlwollend betrachtet, Verbesserungstipps in Bezug auf die Arbeitsweise und Best-Practice an die Mitarbeiter kommunizieren können. Jedoch liegt hier auch der größte Kritikpunkt an My Analytics. Denn umgekehrt lassen sich durch die eindeutige Zuordnung auch vermeintliche Rückschlüsse über die Produktivität jedes Mitarbeiters anstellen, der durch den mehr oder weniger großen Interpretationsspielraum auch angreifbar wird.\\
Die erhobenen Daten sind allerdings nur wirklich aussagekräftig, wenn für die gsamte Unternehmenskommunikation auf Microsoft-Produkte gesetzt wird. Sobald nicht Microsoft Teams oder Skype für Chat und Videokonferenzen genutzt wird, sondern ein anderer Dienst, entfällt dies folglich der Statistik und verfälscht den Score. Somit ist auch eine spätere Auswertung nicht sinnvoll durchzuführen.\\

Gegenüber der Fachzeitschrift \textit{c't}\footnote{Ausgabe 25/2020} erklärte der Rechtsexperte des Deutschen Gewerkschaftsbundes Bertold Brüchner: \glqq Funktionen, mit denen Unternehmen die Arbeitsgepflogenheiten ihrer Bürobelegschaft detailliert durchleuchten können, widersprechen und verstoßen gegen Persönlichkeitsrechte der Mitarbeiter, Datenschutz und – wenn vorhanden – den Beteiligungsrechten- und -pflichten der Betriebs- oder Personalräte.\grqq{} Laut Brüchners Einschätzung ist ein rechtskonformer Einsatz des Features ausgeschlossen, da dies einer Überwachung am Arbeitsplatz gleich kommt und somit gegen geltendes Persönlichkeitsrecht\footnote{Grundlage: Art. 1 Abs. 1 GG (Menschenwürde), Art. 2 Abs. 1 GG (Freie Entfaltung der Persönlichkeit)} verstößt.

\section{Whatsapp im Unternehmen}


\subsection{Freie Software-Alternativen}
Hier soll auf alternativefreie Software aufmerksam gemacht werden und bereits durchgeführte oder in der Durchführung befindliche Projekte (LiMux, Schleswig-Holstein) und deren Erfolg analysiert werden.