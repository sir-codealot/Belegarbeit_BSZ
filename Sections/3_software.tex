\section{Software}
Im zweiten Teil dieser Arbeit wird auf die Umsetzung des Datenschutzes in Software von US-Unternehmen eingegangen und ob dieser immer auch im Sinne des EU-Rechtes umgesetzt wird beziehungsweise wo es Reibungspunkte gibt. Hierbei wird im Besonderen das Augenmerk auf Produkte von Microsoft oder Google, aber auch Programme anderer Hersteller wie Whatsapp als Messenger oder auch Video-Konferenz Programme wie Zoom, welche momentan durch die Pandemie stark an Relevanz gewonnen haben.

\subsection{Situation im Arbeitsalltag und auftretende Probleme}
Blickt man in die Büros deutscher oder auch internationaler Unternehmen, Ämter und Regierungsinstitutionen wird man wenig Varianz bei der genutzten Software finden. In der Regel kommen Software-Lösungen Von Microsoft in Sachen Betriebssysteme und Bürosoftware zum Einsatz, was sie durch ihre hohe Verbreitung zur Standardsoftware erhebt. Weiter wird durch diesen Umstand eine völlige Abkehr von Microsoft-Produkten erschwert, weil wiederum Hersteller von Spezialsoftware ihre Produkte nur für Windows Betriebssysteme anbieten oder nur Schnittstellen für Microsoft Office bereitstellen.\\
Somit kristallisiert sich bereits das erste Problem heraus: die allgemeine Abhängigkeit gegenüber einem Herstellers. \\

\subsection{Microsoft}
\subsubsection{Windows 10}
Hinzu kommt, dass Microsoft-Produkte, wie auch Produkte anderer Hersteller, sogenannte Telemetriedaten erheben, die Aufschluss über das Nutzerverhalten geben. Diese Daten werden zwar anonymisiert übermittelt, die Sammelwut gerade bei Windows 10 und in den ersten Jahren der Einführung ging oftmals weit über das benötigte Maß hinaus. Womit grundsätzlich schon ein Verstoß ein wesentliches Prinzip der DSGVO stattgefunden hat: das Prinzip der Datensparsamkeit. Bei einem frisch aufgesetzten Windows 10 System sind Dienste standardmäßig aktiviert, die beispielsweise den Standort abfragen, Stimmen aufzeichnen und auswerten, damit verbunden sind Mikrofon und Kamera von Haus aus systemweit und somit für jedes Programm aktiviert, das Benachrichtigungssystem lässt es ebenfalls zu, dass Programme Benachrichtigungen senden und teils sogar mitlesen dürfen,
welche von dritten Programmen gesendet werden (beispielsweise eine Benachrichtigung über eine Email mit digitaler Visitenkarte, auf die wiederum die Kontakte-Anwendung direkten Zugriff hat und diese Daten weiterverarbeiten kann) oder auch der Zugriff auf das Telefonbuch eines verknüpften Telefons. Für alle diese Dienste muss der Nutzer bei bei der Ersteinrichtung zum Deaktivieren aktiv werden (sogenanntes Opt-Out). Wird diese durch einen Systemadministrator vorgenommen kann der Nutzer die Einstellungen allerdings immernoch ändern - zu finden in den Systemeinstellungen im Bereich \glqq Datenschutz\grqq{}. Allerdings kommt auch hinzu, dass viele Nutzer sich gar nicht über die im Hintergrund laufenden Dienste bewusst sind. Für eine datenschutzorientierte Konfiguration von Windows 10 stellt die deutsche Verbraucherschutzzentrale einen umfassenden Leitfaden bereit\footnote{https://www.verbraucherzentrale.de/wissen/digitale-welt/datenschutz/datenschutz-bei-windows-10-erhoehen-12154}. Eine komplette Deaktivierung der Datenübermittlung ist des Weiteren trotzdem nicht möglich. Man bekommt nur die Möglichkeit über die Systemeinstellung im Untermenü \textit{Datenschutz $\rightarrow$ Diagnose und Feedback} die Übermittlung auf \glqq erforderliche Diagnosedateien\grqq{} zu stellen und somit lediglich auf ein Minimum zu reduzieren.\\
Microsofts Datenschutzbestimmungen sind im Übrigen auf allen Unterseiten im Menü \glqq Datenschutz\grqq{} verlinkt. Das Dokument ist sehr umfangreich, kleinteilig und außerordentlich offen beschrieben und vermittelt beim Querlesen den Eindruck, dass alle erdenklich sammelbaren Daten auch gesammelt werden. Immerhin stellt Windows die Möglichkeit bereit, Protokolldateien von übermittelten Daten zu speichern und mittels dem hauseigenen Programm \glqq Diagnostic Data Viewer\grqq{} zu sichten.\\

\subsubsection{Microsoft Office}
Auch Microsoft Office sammelt Daten, die durch die Benutzung der Produkte entstehen. Diese werden über den Windows-Dienst \glqq Office-Telemetrie-Prozessor\grqq{} gesammelt, in einer Datenbank abgelegt und bereitet diese in einer Übersicht für Systemadministratoren auf - das sogenannte \glqq Office-Telemetrie-Dashboard\grqq{}. 


\subsection{Bekannte Verstöße}
Hier werden ein paar bekannte Verstöße aufgezeigt, z. B. von Whatsapp, Facebook, Microsoft (dort speziell Windows und MS Office Suites).

\subsection{Freie Software-Alternativen}
Hier soll auf alternativefreie Software aufmerksam gemacht werden und bereits durchgeführte oder in der Durchführung befindliche Projekte (LiMux, Schleswig-Holstein) und deren Erfolg analysiert werden.