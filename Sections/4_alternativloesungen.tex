\subsection{Alternative Produkte}
Zwar ist Microsoft mit seiner Produktfamilie von Betriebssystemen bis Büro-Standardsoftware klarer Marktführer und augenscheinlich ohne wirkliche Konkurrenz, jedoch existieren bis zu einem gewissen Grad auch ernstzunehmende Alternativen. So gibt es für nahezu jedes Microsoft-Programm ein Gegenstück freier Software. Unter diesem Sammelbegriff versteht man frei zugängliche, meist kostenlose Software von welcher auch der Quellcode öffentlich zugänglich ist und welche auch sowohl privat als auch kommerziell verwendet werden darf.\\
Beispiele wären hier \glqq LibreOffice\grqq{}, welches neben einer Textverarbeitung auch eine Tabellenkalkulation und Präsentations-Software mit sich bringt und somit die 3 gängigsten Office-Programme \glqq Word\grqq{}, \glqq Excel\grqq{} und \glqq Powerpoint \grqq{} ersetzt. Der E-Mail-Client \glqq Outlook\grqq{} kann durch Mozillas \glqq Thunderbird\grqq{} abgelöst werden und Microsofts Browser \glqq Internet Explorer\grqq{} und \glqq Edge\grqq{} finden in Form von \glqq Firefox\grqq{} oder \glqq Chromium\grqq{} \footnote{Basis von Googles Chrome Browser, komplett frei von proprietären Softwarekomponenten} adäquaten Ersatz.\\
Will man den Weg zuende gehen kann man sich sogar für ein freies Betriebssystem in Form einer der unzähligen Linux-Distributionen und -Derivate entscheiden. Die bekanntesten und ausgereiftesten Vertreter für den Produktiveinsatz stellen hier \glqq Red Hat Enterprise Linux\grqq{} von Red Hat bzw. dessen Community-Variante \glqq Fedora Linux\grqq{} oder auch \glqq Ubuntu\grqq{} von der Canonical Foundation dar. Red Hat und Canonical bieten einen bezahlten Support für Ihre Betriebssysteme an. Prinzipiell kann man sie aber auch ohne diesen nutzen, dies bringt allerdings auch einen erhöhten Wartungsaufwand mit sich, da bei auftretenden Problemen die Unterstützung seitens des Herstellers fehlt.

\subsection{Kompatibilität}
Grundsätzlich sind die genannten freien Programme zu Microsoft-Dateiformaten kompatibel. Allerdings kann es immer wieder zu Problemen kommen, möchte man beispielsweise ein MS-Word Dokument oder eine Excel-Tabelle mit LibreOffice öffnen. Meist sind es Aspekte der Formatierung, aber auch Funktionsnamen in Excel-Tabellen, die sich unterscheiden können.

\subsection{Das LiMux-Projekt}
\glqq LiMux\grqq\footnote{Kofferwort aus \textbf{Li}n\textbf{ux} und \textbf{M}ünchen} war ein Projekt der Stadtverwaltung München dessen Betreben darin bestand, die gesamte Stadtverwaltung (ca. 15.000 Arbeitsplätze) auf freie Software umzustellen und welches 2013 als abgeschlossen erklärt wurde. Das Projekt galt insgesamt als Erfolg, trotz kleineren anfänglichen Schwierigkeiten, welche aber alle als behebbar galten.\\
Dennoch beschloß die mittlerweile gewechselte Stadtregierung in 2017, die Arbeitsplätze bis 2020 wieder komplett auf Windows zu migrieren. Zum einen dafür verantwortlich wurden Entscheidungen seitens Microsoft, die deutsche Firmenzentrale nach Schwabingen zu verlegen und somit München höhere Steuereinnahmen zu bescheren, zum anderen der Ruf des damaligen Münchener Oberbürgermeisters Dieter Reiter, ein \glqq Microsoft-Fan\grqq{}\footnote{Markus Feilner (07.07.2014), "Microsoft-Fan": Münchens neuer OB Reiter will in Sachen Limux "neue Lösung finden", aufgerufen am 02.04.2021, von https://www.linux-magazin.de/news/microsoft-fan-muenchens-neuer-ob-reiter-will-in-sachen-limux-neue-loesung-finden/} zu sein oder eine durch Microsoft in Auftrag gegebene Studie von HP, welche eine falsche Kostenberechnung des Projektes nahelegt.\footnote{Oliver Diedrich (28.01.2013), Microsoft veröffentlicht Studie zur Linux-Migration in München – teilweise, aufgerufen am 02.04.2021, von https://www.heise.de/newsticker/meldung/Microsoft-veroeffentlicht-Studie-zur-Linux-Migration-in-Muenchen-teilweise-1792252.html} Allerdings warnte die \glqq Free Software Foundation Europe e. V.\grqq{} selbstkritisch davor, die Schuld nur bei Microsoft zu suchen. Auch die hohe Fragmentierung in der städtischen IT, ein schlechtes Projektmanagement und unzureichende interne Kommunikation innerhalb der bayerischen Behörden trugen zur Abkehr von LiMux bei.\footnote{Nick Heath (23.11.2017), From Linux to Windows 10: Why did Munich switch and why does it matter?, aufgerufen am 02.04.2021, von https://www.techrepublic.com/article/linux-to-windows-10-why-did-munich-switch-and-why-does-it-matter/}\\
Bis 2012 sparte die Stadt München durch das Projektes ca. 15,52 Mio Euro ein, Lizenzkosten für notwendige Software-Updates, die alle drei Jahre aufzuwenden sind, nicht mit einberechnet.\footnote{Mirko Dölle (28.03.2012), LiMux: Billiger und robuster als Windows, aufgerufen am 02.04.2021, von https://www.heise.de/newsticker/meldung/LiMux-Billiger-und-robuster-als-Windows-1485410.html}
\vfill