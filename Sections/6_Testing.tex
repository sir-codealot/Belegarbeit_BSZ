Während der Entwicklung wurden die umgesetzten Funktionen bereits getestet, sodass nach der Implementierungsphase nur noch abschließende Gesamt- und Nutzertests durch die Mitarbeiter der Fachabteilung nötig waren.

\subsection{Vorgehen}
    Neue Funktionen wurden während der Implementierung bereits auf viele Szenarien hin eingehend getestet. Zum Beispiel wurden API-Tests durchgeführt, in denen die korrekte Verarbeitung von übergebenen Daten und auch das Verhalten überprüft wurde, wenn Daten nicht in der erwarteten Struktur oder in einem anderen Datentyp vorlagen. Somit konnte sichergestellt werden, dass die Anwendung nicht durch falsche Eingaben zum Absturz gebracht wird, sondern ein entsprechendes Error-Handling greift. Ist die korrekte Funktionsweise eines Anwendungsfalls sichergestellt worden, konnte darauf aufbauend weiterentwickelt werden.

\subsection{Code-Review}
    Das Tool wird im Rahmen einer Code Review von den betreuenden Entwicklern auf Schwachstellen im Code und Einhalten der internen wie externen Coding-Guides hin überprüft. Damit wird eine angestrebte Qualität des Codes gewährleistet.

\subsection{Anwender-Tests}
    Zum aktuellen Zeitpunkt sind noch keine Anwender-Tests erfolgt. Diese sind jedoch als Blackbox-Tests geplant, in denen Mitarbeiter der Fachabteilungen das im Rahmen des Projektes umgesetzte Tool auf dessen Funktionalität hin testen. Dabei haben diese keine Kenntnis des Codestandes.

\subsection{Test-Szenarien}
    Eine Auflistung durchgeführter Test-Szenarien ist im Anhang [NUMMER] zu finden. Daraus ist zu entnehmen, welche Anwendungsfälle vorgesehen waren und deren Testergebnisse.