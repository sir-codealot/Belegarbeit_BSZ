Datenschutz ist sowohl im privaten als auch im Arbeitsumfeld wichtig. Dabei sind die Grenzen beider Bereiche jedoch meist fließend. Denn zum einen verarbeitet man gegebenenfalls selbst personenbezogene Daten Dritter in seinem Beruf, zum anderen werden auch die eigenen Informationen permanent bei der Nutzung verschiedenster Programme oder auch in und durch Institutionen verarbeitet. Dabei kommt es immer wieder vor, dass die Daten nicht datenschutzkonform verarbeitet werden oder dass das dafür zuständige Personal nicht ausreichend sensibilisiert oder geschult ist.\\

Unter dem Begriff Datenschutz versteht man den Schutz der eigenen personenbezogenen Daten und das grundlegende Recht über die Preisgabe und Verarbeitung dieser Daten selbst zu bestimmen. In Deutschland wird dies durch das Grundrecht auf informationelle Selbstbestim-mung\footnote{gestützt auf Artikel 1 \& 2 Grundgesetz} festgeschrieben. Das bedeutet konkret, dass jede natürliche Person alleinig über die Erhebung sie betreffender personenbezogener Daten zustimmen oder auch widersprechen kann. Ein Widerspruch kann zu jeder Zeit, auch nach vorangegangener Erlaubniserteilung, erfolgen. Somit hat jede natürliche Person zu jeder Zeit die uneingeschränkte Handhabe über die sie betreffenden Daten.

\subsection{Personenbezogene Daten}
    Unter personenbezogenen Daten versteht man sämtliche Daten, die einer durch sie identifizierten oder identifizierbarbaren natürlichen Person zugeordnet sind. Darunter zählen alltägliche Daten wie der eigene Name, die eigene Adresse, das Geburtsdatum, Kennnummern wie die Personalausweis -, Sozialversicherungs - oder Kontonummer, Online-Kennungen wie Login- und Nicknames,Standortdaten und IP-Adressen oder die eigene Telefonnummer.\\
    Aber auch Daten, die über den physischen, physiologischen oder psychischen Zustand, die eigene Krankheitshistorie, sexuelle Orientierung oder die politische Einstellung Aufschluss geben, fallen ebenfalls unter diesen Sammelbegriff, werden aber aufgrund ihrer besonderen Sensibilität als \glqq besondere personenbezogene Daten\grqq{} speziell behandelt. Ihrer Verarbeitung muss stets separat zugestimmt werden.

\subsection{Europäische Rechtslage}
    In Europa gilt seit Mai 2018 in Form der \glqq Datenschutzgrundverordnung\grqq{} eine für alle Mitgliedstaaten der Europäischen Union bzw. Staaten des europäischen Wirtschaftsraumes verbindliche Verordnung, die den Datenverkehr innerhalb dieser regelt. Diese wird durch die jeweilige nationale Gesetzgebung der einzelnen Länder ergänzt, in Deutschland in Form des Bundesdatenschutzgesetzes.\\
    Dabei darf die Datenschutzgrundverordnung auf nationaler Ebene grundsätzlich weder verschärft noch abgeschwächt werden. Sie bietet allerdings auch sogenannte Öffnungsklauseln, durch die den Mitgliedsstaaten in den betreffenden Punkten eine relativ offene Auslegungshoheit der eigenen Gesetze gewährt wird. Somit kann man die DSGVO auch als Hybrid aus Richtlinie und Verordnung verstehen.


\subsection{Vergleich von US- und EU-Datenschutz}
    Der Datenschutz in den vereinigten Staaten ist für uns Europäer besonders dann interessant, wenn es um große IT-Konzerne wie Microsoft, Google, Facebook oder Anbietern für Video-Telefonie wie Zoom oder Discord geht, die derzeit durch die aktuellen Gegebenheiten stark an Relevanz gewonnen haben. In der öffentlichen Diskussion macht es bisweilen den Anschein, dass in den USA eine sehr lockere Form bis gar kein Datenschutz vorhanden ist. Prinzipiell ist es so, dass der Datenschutz in Übersee brachenspeziefisch geregelt ist. Ähnlich wie in der EU sind Unternehmen in den USA dazu verpflichtet, die Sicherheit personenbezogener Daten zu garantieren und unterliegen einer Meldepflicht, falls Datenlecks auftreten. Das jeweilige Datenschutzniveau wird hingegen von jedem Unternehmen selbst festgelegt und unterliegt dadurch keiner zentralisierten Richtlinie oder Verordnung, wie es durch die europäische Rechts-sprechung der Fall ist. Des Weiteren gibt es auch eine Aufsichtsbehörde, die die Unternehmen kontrolliert, sie auf Nachbesserung drängen und gegebenenfalls hohe Sanktionen verhängen kann.\\
    Seit August 2020 hat der Staat Kalifornien einen \glqq europäischen Weg\grqq{} mit einer restriktiveren Datenschutzverordnung eingeschlagen\footnote{California Consumer Privacy Act}. Dadurch, dass viele großen Tech-Firmen ihren Hauptsitz in diesem Bundesstaat haben, fallen sie auch unter dessen Rechtssprechung. Auch andere Bundesstaaten wie Etwa New York sind derzeit dabei, ähnliche Gesetzesgrundlagen zu schaffen.\\
    Bei der Umsetzung des Datenschutzes verfolgen beide Rechtssysteme einen grundlegend anderen Ansatz. Während in der EU der Schutz personenbezogener Daten ein Grundrecht darstellt, ist er in den USA Bestandteil des Verbraucherschutzrechtes. Dies wird auch dadurch unterstrichen, dass für die Aufsicht und Kontrolle der Unternehmen hinsichtlich deren Einhaltung des Datenschutzes die Bundeshandelskommission\footnote{Federal Trade Commission (FTC)} der USA zuständig ist.\\
    Ein weiterer zu berücksichtigender Aspekt ist, dass personenbezogene Daten in den USA auch von staatlicher Seite und ohne Berücksichtigung der im einzelnen geltenden Gesetze ausgewertet werden können. Die Grundlage dafür stellt der \glqq USA PATRIOT Act\grqq{}\footnote{\textbf{U}niting and \textbf{S}trengthening \textbf{A}merica by \textbf{P}roviding \textbf{A}ppropriate \textbf{T}ools \textbf{R}equired to \textbf{I}ntercept and \textbf{O}bstruct \textbf{T}errorism Act of 2001} dar, der es erlaubt eben diese Daten zur Terrorismusbekämpfung von Unternehemen abzurufen und zu verarbeiten - und das komplett ohne vorliegenden gerichtlichen Beschluss. Somit sind potentiell alle personenbezogenen Daten, die von Unternehmen in den USA verarbeitet werden auch der Regierung frei zugänglich. Ein unter europäischem Recht nicht denkbarer Umstand. Seit den Enthüllungen durch Edward Snowden ist zudem bekannt, dass Auswertungen von gespeicherten Daten auf in den USA befindlichen Servern nicht nur punktuell, sondern permanent und flächendeckend stattfinden.\\
%\vfill
%\pagebreak
\subsection{Vereinbarkeit von EU- und US-Recht}
    Ab 1995 schreibt die EU-Datenschutzrichtlinie\footnote{Richtlinie 95/46/EG, 2018 durch DSGVO abgelöst} vor, dass personenbezogene Daten nicht an unsichere Drittländer übermittelt werden dürfen. Mit ihrer aktuellen Rechtssprechung fallen die USA unter eben diese Definition, ein Datenaustausch seitens der europäischen Union ist somit rechtlich nicht vertretbar.\\
    Um dem entgegen zu wirken gab es mehrere Bestrebungen, Sonderregelungen zu treffen und somit eine weitere Zusammenarbeit zu ermöglichen.

\subsubsection{Safe Harbor Abkommen}
    Bereits im Jahr 2000 beschlossen stellt es den ältesten Versuch dar, geltendes EU- und US-Datenschutz in Einklang zu bringen. Das Abkommen sollte es ermöglichen, personenbezogene Daten aus einem Mitgliedsland der EU in die USA übermitteln zu dürfen. Der Europäische Gerichtshof erklärte dieses jedoch am 6. Oktober 2015 für ungültig.

\subsubsection{Privacy Shield}
    Ab August 2016 trat die Nachfolgeregelung für Safe Harbor in Kraft. Sie sollte das für nichtig erklärte Vorgängerabkommen ersetzen und grundsätzlich die selbe Funktion wie dieses übernehmen. Jedoch wurde auch das Privacy Shield vom EuGH am in einem Urteil vom 16. Juli 2020 für ungültig erklärt. Aktuell ist noch kein vergleichbares Abkommen in Kraft getreten, welches den Privacy Shield ersetzt.
\vfill