\subsection{IST-Zustand}
    Hier wird der Stand der Dinge erläutert.

\subsection{Datenschutz}
    Was allgemeines über den bestehenden Datenschutz.

\subsection{DSGVO und BDSG}
    Hier geht es um die europäischen Richtlinien in Bezug auf den Datenschutz (Datenschutzgrundverordung) und die nationale, ergänzende Gesetzgebung (Bundesdatenschutzgesetz).

\subsection{Vergleich von US- und EU-Datenschutz}
    Hier sollen die Unterschiede nach beiden Rechtslagen aufgezeigt werden.

\subsection{Probleme mit US-Software und deren Nutzung in Unternehmen}
    Der Kern der Belegarbeit. Hier wird abgehandelt, inwiefern Software von US-Unternehmen den europäischen Datenschutzstandards entspricht, wo es Reibungspunkte gibt und wo klare Verstöße vorliegen.

\subsection{Bekannte Verstöße}
    Hier werden ein paar bekannte Verstöße aufgezeigt, z. B. von Whatsapp, Facebook, Microsoft (dort speziell Windows und MS Office Suites).

\subsection{Freie Software-Alternativen}
    Hier soll auf alternativefreie Software aufmerksam gemacht werden und bereits durchgeführte oder in der Durchführung befindliche Projekte (LiMux, Schleswig-Holstein) und deren Erfolg analysiert werden.