Bei der Projektplanung wurde ein passendes Vorgehensmodell gewählt, der Projektablauf in Phasen unterteilt und die benötigten Ressourcen geplant.

\subsection{Vorgehensmodell}
    Es wurde sich für das iterative Modell entschieden, da innerhalb des Entwicklungsprozesses Funktionen parallel implementiert und getestet wurden. Das Vorgehensmodell beinhaltet, dass Phasen mehrmals durchlaufen werden und innerhalb jeder Iteration die Software geändert, weiterentwickelt und somit erweitert wird. Hierdurch werden zwischenzeitliche Tests während der Entwicklung ermöglicht und die Erweiterung der Software schrittweise auf Grundlage bereits umgesetzter Funktionalitäten vorangetrieben.

\subsection{Projektphasen}
    Für die Umsetzung des Projektes wurde eine Dauer von 120 Arbeitsstunden vorgegeben. Diese konnten frei innerhalb des Projektzeitraumes von Projektbeginn (24. September 2021) bis Projektabgabe (19. November 2021) für die Umsetzung aufgewendet werden. Es erfolgte eine zeitliche Einteilung in die Phasen Analyse und Konzeption, Implementierung und Testen, Release-Testing, Übergabe und Verfassen der Dokumentation. Eine detaillierte zeitliche Gliederung ist in Anlage [NUMMER] zu finden.\\
    Innerhalb des Projektes gab es 5 Meilensteine, namentlich das \glqq Kickoff-Meeting\grqq{}, \glqq Abschluss Analysephase\grqq{}, \glqq Abschluss Implementierungsphase\grqq{}, \glqq Abschluss Testphase\grqq{}, die jeweils am Ende der jeweiligen Projektphasen erreicht wurden. Der Meilenstein \glqq Projektabschluss\grqq{} stellt auch zugleich die Beendigung des Projektes dar. Die für das Projekt verfügbaren Arbeitsstunden wurden wie folgt auf die unterschiedlichen Phasen aufgeteilt:\\

    \vspace{1cm}
    \begin{table}[h]
        \centering
        \begin{tabular}{l|c}
            Phase & Zeit in Stunden \\
            \hline
            Analysephase & 10 \\
            Implementierung & 60 \\
            Testphase & 30 \\
            Projektabschluss (Übergabe und Dokumentation) & 20
        \end{tabular}
        \caption{Zeitplanung}
    \end{table}
\pagebreak
\subsection{Ressourcenplanung}
    Für dieses Projekt wurde ausschließlich bereits im Unternehmen vorhandene, lizenzierte oder freie Software sowie Hardware genutzt, wodurch keine Neuanschaffungen im Projektrahmen notwendig sind. Eine genaue Aufschlüsselung befindet sich in Anlage [Nummer]. Das Projekt wurde von dem Auszubildenden an einem vorhandenden PC-Arbeitplatz durchgeführt. Zur Betreuung und für Rückfragen standen insgesamt 3 Entwickler zur Seite.\\
    Das Projekt wurde unter Zuhilfenahme der Versionsverwaltung Git bzw. der Versionierungsplatform Gitlab umgesetzt, um eine adäquate Codeverwaltung und fachgerechte Verteilung zu gewährleisten.

\vfill