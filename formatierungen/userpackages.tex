% \usepackage{xcolor}
% \usepackage{lingmacros}
\usepackage{tree-dvips}
\usepackage[utf8]{inputenc}
\usepackage{fancyhdr}
\usepackage[ngerman]{babel}
\usepackage{graphicx}
\usepackage{wrapfig,lipsum}
\usepackage[tocflat]{tocstyle}
\usepackage{tocloft}
\usepackage[perpage]{footmisc}
\usepackage[ngerman, num]{isodate}
% \usepackage{fourier}
\usepackage{caption}
\usepackage[parfill]{parskip}
\PassOptionsToPackage{hyphens}{url}\usepackage{hyperref}
% % * Set page layout
% % Seitenränder
\usepackage[a4paper,left=2.5cm,right=2.5cm,top=2cm,bottom=2cm]{geometry}
\usepackage{csvsimple}
\usepackage{helvet}
\renewcommand{\familydefault}{\sfdefault}
\usepackage{enumerate}
\usepackage{enumitem}

%subsubsubsections
%\usepackage{titlesec}
%\setcounter{secnumdepth}{4}
%\setcounter{tocdepth}{4}
%
%\titleformat{\paragraph}
%{\normalfont\normalsize\bfseries}{\theparagraph}{1em}{}
%\titlespacing*{\paragraph}
%{0pt}{3.25ex plus 1ex minus .2ex}{1.5ex plus .2ex}
%
%\newcommand{\subsubsubsection}[1]{\paragraph{#1}}



% Kopfzeile
\linespread{1.25}
\usepackage{fancyhdr}
\setlength{\headheight}{14pt}
\pagestyle{fancy}
% ------------------------------

% Seitenzahl in Fusszeile rechts
\fancyfoot{}
\fancyfoot[R]{\thepage}
% ------------------------------

\renewcommand{\refname}{}
\patchcmd{\thebibliography}{\section*{\refname}}{}{}{}


% Bilder und Bildunterschriften ---------------------
% Bilder Referenzen
\usepackage{graphicx}
\graphicspath{ {bilder/} }

% Bilduntertitel-Format "blank" fuer leere Captions
%\DeclareCaptionLabelFormat{blank}{}

% Titel ohne babel ...
%\renewcommand{\listfigurename}{}

% Titel mit babel und ngerman als Sprache
\addto\captionsngerman{%
  \renewcommand{\listfigurename}%
    {}%
}

\makeatother
\renewcommand{\figurename}{Abbildung}
%\renewcommand{\thefigure}{\thefigure \arabic{figure}}
\renewcommand{\cftfigfont}{Abb. }
% -----------------------------------------------------

% Inhaltsverzeichnis ------------------------------------------------
% Titel ohne babel ...
%\renewcommand{\contentsname}{Inhalt}

% ... und mit babel und ngerman als Sprache
\addto\captionsngerman{% Replace "ngerman" with the language you use
  \renewcommand{\contentsname}%
    {Inhalt}%
}

%\renewcommand{\cftsecleader}{\cftdotfill{\cftdotsep}}
% -------------------------------------------------------------------

%\renewcommand{\cftsecfont}{\normalfont\slshape\}


\usepackage{booktabs}% http://ctan.org/pkg/booktabs
\newcommand{\tabitem}{~~\llap{\textbullet}~~}

\usepackage{makecell}
\usepackage{longtable}

\newcommand*{\source}[3]{%
  \caption[#1 \newline #2 \newline #3]{#1}
}

\usepackage{xltabular}

\usepackage{float}

\renewcommand{\listtablename}{}
\renewcommand{\tablename}{Tabelle}

\setlength{\parindent}{0pt}

% Abstand zwischen Text und horizontaler Ebene über Fußnoten
\setlength{\skip\footins}{1cm}

\newcommand{\headingSpace}{1.5cm}

% Für die Einrückung wird das Paket tocloft benötigt
%\cftsetindents{chapter}{0.0cm}{\headingSpace}
\cftsetindents{section}{0.0cm}{\headingSpace}
\cftsetindents{subsection}{0.0cm}{\headingSpace}
\cftsetindents{subsubsection}{0.0cm}{\headingSpace}